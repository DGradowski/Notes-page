\documentclass{article}
\usepackage[utf8]{inputenc}
\usepackage[T1]{fontenc}
\usepackage{xcolor}
\usepackage{fontspec}
\setmainfont{Book Antiqua}
\usepackage{setspace} % do odstępów
\usepackage{quoting} % ładne cytaty
\usepackage{amsmath}
\usepackage{multirow}
\usepackage{array}

\newcolumntype{M}[1]{>{\centering\arraybackslash}m{#1}}

% Ustawienia kolorów
\pagecolor{black}
\color{white}
\newcommand{\zero}{\textcolor{red}{0}}
\newcommand{\one}{\textcolor{green}{1}}

\begin{document}

% Strona tytułowa
\begin{titlepage}
    \centering
    {\Huge\bfseries Egzamin tele\par}
    \vspace{2cm}
    {\Large Dawid Gradowski \par}
    {\large PuckMoment \par}
    \vfill

    \begin{quoting}[]
        \begin{center}
            \Large„In this world, is the destiny of mankind controlled by some transcendental entity or law? Is it like the hand of God hovering above?”
        \end{center}
    \end{quoting}
    
    \vfill
    {\large 28.06.2025r.\par}
\end{titlepage}
\tableofcontents
\section*{Prolog}
Wszystkie notatki robione na podstawie tego co było na pierwszym terminie egzaminu.
Wszystkie przykłady zapamiętałem i przepisałem na kartkę, nigdy nie doszło do żadnego nielegalnie
zrobionego zdjęcia.
\section{Dyskretne przekształcenie Fouriera ciągu $f(t)$}
Przykładowe zadanie:
\[
    f(t) = 2 \cos(2\omega_0 t) + 3 \cos(4\omega_0 t) + 7 \cos(5 \omega_0 t)
\]
W tym zadaniu trzeba zapamiętać że:
\[
    \cos(\alpha t) \longleftrightarrow \pi(\delta (\omega - \alpha) + \delta (\omega + \alpha))
\]
\[
    \sin(\alpha t) \longleftrightarrow j \pi(\delta (\omega + \alpha) - \delta (\omega - \alpha))
\]
gdzie $\alpha$ to jakaś wartość.

Możemy sobie rozbić wzór naszej funkcji na części i zastosować wyżej wymienione reguły:
\begin{enumerate}
    \item {
        \[
            2\cos(2\omega_0 t) \longleftrightarrow 2\pi(\delta (\omega - 2\omega_0) + \delta (\omega + 2\omega_0))
        \]
    }
    \item {
        \[
            3\cos(4\omega_0 t) \longleftrightarrow 3\pi(\delta (\omega - 4\omega_0) + \delta (\omega + 4\omega_0))
        \]
    }
    \item {
        \[
            7\cos(5\omega_0 t) \longleftrightarrow 7\pi(\delta (\omega - 5\omega_0) + \delta (\omega + 5\omega_0))
        \]
    }
\end{enumerate}
Teraz wystarczy to wszystki zsumować:
\[
    F(\omega) = 2\pi(\delta (\omega - 2\omega_0) + \delta (\omega + 2\omega_0)) + 3\pi(\delta (\omega - 4\omega_0) + \delta (\omega + 4\omega_0)) + 7\pi(\delta (\omega - 5\omega_0) + \delta (\omega + 5\omega_0))
\]


\section{Dyskretne przekształcenie Fouriera ciągu $x[n]$}
Przykładowe zadanie:
\[
    x[n]=[2;3;1;2]
\]
Dla syngału dyskretnego $x[n]$, $n = 0,1,...,N-1$, DFT jest zdefiniowane jako:
\[
    X[k]=\sum_{n=0}^{N-1}x[n] \cdot e^{-j \frac{2 \pi}{N} k n}\text{, dla } k = 0,1,...,N - 1
\]
do tego warto znać jeszcze wzór Eulera, z którego mamy:
\[
    e^{j \alpha} = \cos(\alpha) + j \sin(\alpha)
\]
\[
    e^{-j \alpha} = \cos(\alpha) - j \sin(\alpha)
\]
w naszym przykładzie $\alpha = \frac{2 \pi}{N} k n$. Dlatego powyższy wzór możemy zapisać w następujący sposób:
\[
    X[k]=\sum_{n=0}^{N-1}x[n] \cdot e^{-j \frac{2 \pi}{N} k n} = \sum_{n=0}^{N-1}x[n] \cdot (cos(\frac{2 \pi}{N} k n) - j \sin(\frac{2 \pi}{N} k n))
\]

Dla naszego przykładu:
\[
    x[n]=[2;3;1;2] \text{ oraz } N = 4
\]
obliczamy
\[
    X[k]=\sum_{n=0}^{3}x[n] \cdot e^{-j \frac{2 \pi}{4} k n}\text{, dla } k = 0,1,2,3
\]
\[
    X[0]=\sum_{n=0}^{3}x[n] \cdot e^{-j \frac{2 \pi}{4} 0 n} =
    2 + 3 + 1 + 2 = 8
\]
\[
    X[1]=\sum_{n=0}^{3}x[n] \cdot e^{-j \frac{2 \pi}{4} 1 n} = \sum_{n=0}^{3}x[n] \cdot (\cos(\frac{2 \pi}{4}n) - j \sin(\frac{2 \pi}{4}n)) = 
\]
\[
    2 + 3 \cdot (\cos(\frac{2 \pi}{4} 1) - j \sin(\frac{2 \pi}{4} 1)) + (\cos(\frac{2 \pi}{4} 2) - j \sin(\frac{2 \pi}{4} 2))
    + 2 \cdot (\cos(\frac{2 \pi}{4} 3) - j \sin(\frac{2 \pi}{4} 3)) = 
\]
\[
    2 + 3(\cos(\frac{\pi}{2}) - j\sin(\frac{\pi}{2})) + (\cos(\pi) - j \sin(\pi)) + 2(\cos{\frac{3\pi}{2}} - j\sin(\frac{3\pi}{2})) = 
\]
\[
    2 - 3j - 1 + 2j = 1 - 1j
\]

\[
    X[2]=\sum_{n=0}^{3}x[n] \cdot e^{-j \frac{2 \pi}{4} 2 n} = 
    \sum_{n=0}^{3}x[n] \cdot (\cos(\frac{2 \pi}{4}2n) - j \sin(\frac{2 \pi}{4}2n)) = 
\]
\[
    2 + 3 \cdot (\cos(\frac{2 \pi}{4} 2) - j \sin(\frac{2 \pi}{4} 2)) + (\cos(\frac{2 \pi}{4} 4) - j \sin(\frac{2 \pi}{4} 4))
    + 2 \cdot (\cos(\frac{2 \pi}{4} 6) - j \sin(\frac{2 \pi}{4} 6)) = 
\]
\[
    2 + 3(\cos(\pi) - j\sin(\pi)) + (\cos(2\pi) - j \sin(2\pi)) + 2(\cos{(3\pi)} - j\sin(3\pi)) = 
\]
\[
    2 + 3(\cos(\pi) - j\sin(\pi)) + (\cos(0) - j \sin(0)) + 2(\cos{(\pi)} - j\sin(\pi)) = 
\]
\[
    2 - 3 + 1 - 2 = -2
\]
\newpage
\[
    X[3]=\sum_{n=0}^{3}x[n] \cdot e^{-j \frac{2 \pi}{4} 3 n} = 
    \sum_{n=0}^{3}x[n] \cdot (\cos(\frac{2 \pi}{4}3n) - j \sin(\frac{2 \pi}{4}3n)) = 
\]
\[
    2 + 3 \cdot (\cos(\frac{2 \pi}{4} 3) - j \sin(\frac{2 \pi}{4} 3)) + (\cos(\frac{2 \pi}{4} 6) - j \sin(\frac{2 \pi}{4} 6))
    + 2 \cdot (\cos(\frac{2 \pi}{4} 9) - j \sin(\frac{2 \pi}{4} 9)) = 
\]
\[
    2 + 3(\cos(\frac{3\pi}{2}) - j\sin(\frac{3\pi}{2})) + (\cos(3\pi) - j \sin(3\pi)) + 2(\cos{\frac{9\pi}{2}} - j\sin(\frac{9\pi}{2})) = 
\]
\[
    2 + 3(\cos(\frac{3\pi}{2}) - j\sin(\frac{3\pi}{2})) + (\cos(\pi) - j \sin(\pi)) + 2(\cos{\frac{\pi}{2}} - j\sin(\frac{\pi}{2})) = 
\]
\[
    2 + 3j - 1 - 2j = 1 + 1j
\]
więc odpowiedź to:
\[
    X[n] = [8, 1 - 1j, -2, 1 + 1j]
\]

Jeśli ktoś ma problem z zapamiętaniem wartości trygonometrycznych to proponuje zapamiętać taką tabelke:
\begin{center}
\renewcommand{\arraystretch}{1.5}
\begin{tabular}{|M{1cm}|M{1cm}|M{1cm}|M{1cm}|M{1cm}|M{1cm}|M{1cm}|}
    \hline
    $\alpha$ & 0 & $\frac{\pi}{6}$ & $\frac{\pi}{4}$ & $\frac{\pi}{3}$ & $\frac{\pi}{2}$ & $\pi$ \\
    \hline
    $\sin(\alpha)$ & $\frac{\sqrt{0}}{2}$ & $\frac{\sqrt{1}}{2}$ & $\frac{\sqrt{2}}{2}$ & $\frac{\sqrt{3}}{2}$ & $\frac{\sqrt{4}}{2}$ & 0 \\
    \hline
    $\cos(\alpha)$ & $\frac{\sqrt{4}}{2}$ & $\frac{\sqrt{3}}{2}$ & $\frac{\sqrt{2}}{2}$ & $\frac{\sqrt{1}}{2}$ & $\frac{\sqrt{0}}{2}$ & -1 \\
    \hline
\end{tabular}
\end{center}
Co zapisane w normalny sposób daje
\begin{center}
\renewcommand{\arraystretch}{1.5}
\begin{tabular}{|M{1cm}|M{1cm}|M{1cm}|M{1cm}|M{1cm}|M{1cm}|M{1cm}|}
    \hline
    $\alpha$ & 0 & $\frac{\pi}{6}$ & $\frac{\pi}{4}$ & $\frac{\pi}{3}$ & $\frac{\pi}{2}$ & $\pi$ \\
    \hline
    $\sin(\alpha)$ & $0$ & $\frac{1}{2}$ & $\frac{\sqrt{2}}{2}$ & $\frac{\sqrt{3}}{2}$ & $1$ & 0 \\
    \hline
    $\cos(\alpha)$ & $1$ & $\frac{\sqrt{3}}{2}$ & $\frac{\sqrt{2}}{2}$ & $\frac{{1}}{2}$ & $0$ & -1 \\
    \hline
\end{tabular}
\end{center}
\section{Filtracja sygnału}
Przykładowe zadanie:
\[
    \text{Sygnał: } x[n]=[4;1;7;3]
\]
\[
    \text{Filtr: } h[n]=[2;4;1]
\]
Długość przefiltrowanego sygnału to suma długości sygnału i filtra minus 1.
\[
    len(x) + len(h) - 1 = len(y)
\]
\[
    4 + 3 - 1 = 6
\]

Wzór ogólny przy ograniczonym zakresie wygląda następująco:
\[
    y[n] = \sum_{k=0}^{M - 1}x[k] \cdot h[n - k]
\]
gdzie, $M$ - długość x oraz zakładamy, dla wartości x i h spoza zakresu przyjmujemy 0.

Obliczny sobie teraz nasz przykład:
\[
    y[0] = x[0] \cdot k[0 - 0] = 4 \cdot 2 = 8
\]
\[
    y[1] = x[0] \cdot k[1 - 0] + x[1] \cdot k[1 - 1] = 4 \cdot 4 + 1 \cdot 2 = 18
\]
\[
    y[2] = x[0] \cdot k[2] + x[1] \cdot k[1] + x[2] \cdot k[0]  = 4 \cdot 1 + 1 \cdot 4 + 7 \cdot 2 = 22
\]
\[
    y[3] = x[1] \cdot k[2] + x[2] \cdot k[1] + x[3] \cdot k[0] = 1 \cdot 1 + 7 \cdot 4 + 3 \cdot 2 = 35
\]
\[
    y[4] = x[2] \cdot k[2] + x[3] \cdot k[1] = 7 \cdot 1 + 3 \cdot 4 = 19
\]
\[
    y[5] = x[3] \cdot k[2] = 3 \cdot 1  = 3
\]
A więc otrzymujemy
\[
    y[n] = [8,18,22,35,19,3]
\]
\section{Kod Hamminga}
Przykładowe zadanie:
\[
    x[n]=[1,0,1,0,0,1]
\]
na 10 bitach
\subsection*{Rozwiązanie}
Kod Hamminga pozwala nam na poprawienie jednego uszkodzonego bitu w wiadomości
o długości $p + m$, gdzie: $m$ oznacza ilość bitów oryginalnej wiadomości, a $p$ ilość
bitów parzystości w kodzie. W naszym przypadku mamy więc:
\[
    m = 6 \text{ oraz } p = 4
\]
Ważne jest, że:
\[
    2^p \ge (p + m) + 1
\]
W naszym przypadku $2^4$ jest większe niż $(6 + 4) + 1$ więc kod Hamminga będzie działał.

Ważne jest rozmieszczenie bitów parzystości. Za pomocą $p_x$ przedstawimy bity
parzystości, a za pomocą $m_x$ bity wiadomości.

\begin{center}
    \begin{tabular}{|c|c|c|c|c|c|c|c|c|c|c|}
        \hline
        Numer bita & 1 & 2 & 3 & 4 & 5 & 6 & 7 & 8 & 9 & 10 \\
        \hline
        Znaczenie & $p_1$ & $p_2$ & $m_1$ & $p_3$ & $m_2$ & $m_3$ & $m_4$ & $p_4$ & $m_5$ & $m_6$ \\
        \hline
    \end{tabular}
\end{center}

Można zauważyć, że $p_x$ znajduje się na miejscu $2^{x - 1}$. Dlatego $p_1$ znajduje się
na pozycji pierwszej, a $p_4$ na 8.

Bity parzystości oprócz przydzielonego z góry miejsca, mają również bity które im odpowiadają.
Dla bitu $p_1$ będą to $(1, 3, 5, 7, 9)$. Oznacza to, że na tych 5 bitach liczba jedynek musi być
parzysta. Teraz pytanie skąd wziąć ten zbiór? Pierwsza liczba to zawsze jest miejsce
w którym znajduje się bit parzystości. Liczba opisująca który to jest bit parzystości mówi nam co
ile bitów następuje zmiana. Jeśli nie wiesz o co chodzi rozpiszę to ładnie.
\begin{center}
    \begin{tabular}{|c|c|c|c|c|c|c|c|c|c|c|}
        \hline
        Numer bita & 1 & 2 & 3 & 4 & 5 & 6 & 7 & 8 & 9 & 10 \\
        \hline
        $p_1$ & \textcolor{green}{1} & \textcolor{red}{0} & \textcolor{green}{1} & \textcolor{red}{0} & \textcolor{green}{1} & \textcolor{red}{0} & \textcolor{green}{1} & \textcolor{red}{0} & \textcolor{green}{1} & \textcolor{red}{0} \\
        \hline
        $p_2$ & \textcolor{red}{0} & \textcolor{green}{1} & \textcolor{green}{1} & \textcolor{red}{0} & \textcolor{red}{0} & \textcolor{green}{1} & \textcolor{green}{1} & \textcolor{red}{0} & \textcolor{red}{0} & \textcolor{green}{1} \\
        \hline
        $p_3$ & \textcolor{red}{0} & \textcolor{red}{0} & \textcolor{red}{0} & \textcolor{green}{1} & \textcolor{green}{1} & \textcolor{green}{1} & \textcolor{green}{1} & \textcolor{red}{0} & \textcolor{red}{0} & \textcolor{red}{0} \\
        \hline
        $p_4$ & \textcolor{red}{0} & \textcolor{red}{0} & \textcolor{red}{0} & \textcolor{red}{0} & \textcolor{red}{0} & \textcolor{red}{0} & \textcolor{red}{0} & \textcolor{green}{1} & \textcolor{green}{1} & \textcolor{green}{1} \\
        \hline
    \end{tabular}
\end{center}
Mając tą wiedze spróbujmy zakodować wiadomość. Nasza wiadomość to:
\[
    x[n]=[1,0,1,0,0,1]
\]
a więc dodając bity parzystości otrzymamy coś takiego:
\begin{center}
        \begin{tabular}{|c|c|c|c|c|c|c|c|c|c|c|}
        \hline
        Numer bita & 1 & 2 & 3 & 4 & 5 & 6 & 7 & 8 & 9 & 10 \\
        \hline
        Kod & $p_1$ & $p_2$ & 1 & $p_3$ & 0 & 1 & 0 & $p_4$ & 0 & 1 \\
        \hline
    \end{tabular}
\end{center}
Sprawdźmy co powinno być w miejscu $p_1$:
\begin{center}
        \begin{tabular}{|c|c|c|c|c|c|c|c|c|c|c|}
        \hline
        Numer bita & \textcolor{green}{1} & 2 & \textcolor{green}{3} & 4 & \textcolor{green}{5} & 6 & \textcolor{green}{7} & 8 & \textcolor{green}{9} & 10 \\
        \hline
        Kod & \textcolor{green}{$p_1$} & $p_2$ & \textcolor{green}{1} & $p_3$ & \textcolor{green}{0} & 1 & \textcolor{green}{0} & $p_4$ & \textcolor{green}{0} & 1 \\
        \hline
    \end{tabular}
\end{center}
Na pozycjach które psrawdzamy dla $p_1$ mamy na nieparzystą liczbę jedynek dlatego w miejscu 
$p_1$ wpisujemy jeden.
\begin{center}
        \begin{tabular}{|c|c|c|c|c|c|c|c|c|c|c|}
        \hline
        Numer bita & \textcolor{green}{1} & 2 & \textcolor{green}{3} & 4 & \textcolor{green}{5} & 6 & \textcolor{green}{7} & 8 & \textcolor{green}{9} & 10 \\
        \hline
        Kod & \textcolor{green}{1} & $p_2$ & \textcolor{green}{1} & $p_3$ & \textcolor{green}{0} & 1 & \textcolor{green}{0} & $p_4$ & \textcolor{green}{0} & 1 \\
        \hline
    \end{tabular}
\end{center}
Teraz pozycja $p_2$:
\begin{center}
        \begin{tabular}{|c|c|c|c|c|c|c|c|c|c|c|}
        \hline
        Numer bita & 1 & \textcolor{green}{2} & \textcolor{green}{3} & 4 & 5 & \textcolor{green}{6} & \textcolor{green}{7} & 8 & 9 & \textcolor{green}{10} \\
        \hline
        Kod & 1 & \textcolor{green}{$p_2$} & \textcolor{green}{1} & $p_3$ & 0 & \textcolor{green}{1} & \textcolor{green}{0} & $p_4$ & 0 & \textcolor{green}{1} \\
        \hline
    \end{tabular}
\end{center}
Liczba jedynek nieparzysta dlatego w miejscu $p_2$ wpisujemy 1:
\begin{center}
        \begin{tabular}{|c|c|c|c|c|c|c|c|c|c|c|}
        \hline
        Numer bita & 1 & \textcolor{green}{2} & \textcolor{green}{3} & 4 & 5 & \textcolor{green}{6} & \textcolor{green}{7} & 8 & 9 & \textcolor{green}{10} \\
        \hline
        Kod & 1 & \textcolor{green}{1} & \textcolor{green}{1} & $p_3$ & 0 & \textcolor{green}{1} & \textcolor{green}{0} & $p_4$ & 0 & \textcolor{green}{1} \\
        \hline
    \end{tabular}
\end{center}
Teraz pozycja $p_3$:
\begin{center}
        \begin{tabular}{|c|c|c|c|c|c|c|c|c|c|c|}
        \hline
        Numer bita & 1 & 2 & 3 & \textcolor{green}{4} & \textcolor{green}{5} & \textcolor{green}{6} & \textcolor{green}{7} & 8 & 9 & 10 \\
        \hline
        Kod & 1 & 1 & 1 & \textcolor{green}{$p_3$} & \textcolor{green}{0} & \textcolor{green}{1} & \textcolor{green}{0} & $p_4$ & 0 & 1 \\
        \hline
    \end{tabular}
\end{center}
Liczba jedynek nieparzysta dlatego w miejscu $p_3$ wpisujemy 1:
\begin{center}
        \begin{tabular}{|c|c|c|c|c|c|c|c|c|c|c|}
        \hline
        Numer bita & 1 & 2 & 3 & \textcolor{green}{4} & \textcolor{green}{5} & \textcolor{green}{6} & \textcolor{green}{7} & 8 & 9 & 10 \\
        \hline
        Kod & 1 & 1 & 1 & \textcolor{green}{1} & \textcolor{green}{0} & \textcolor{green}{1} & \textcolor{green}{0} & $p_4$ & 0 & 1 \\
        \hline
    \end{tabular}
\end{center}
Teraz pozycja $p_4$:
\begin{center}
        \begin{tabular}{|c|c|c|c|c|c|c|c|c|c|c|}
        \hline
        Numer bita & 1 & 2 & 3 & 4 & 5 & 6 & 7 & \textcolor{green}{8} & \textcolor{green}{9} & \textcolor{green}{10} \\
        \hline
        Kod & 1 & 1 & 1 & 1 & 0 & 1 & 0 & \textcolor{green}{$p_4$} & \textcolor{green}{0} & \textcolor{green}{1} \\
        \hline
    \end{tabular}
\end{center}
Liczba jedynek nieparzysta dlatego w miejscu $p_4$ wpisujemy 1:
\begin{center}
        \begin{tabular}{|c|c|c|c|c|c|c|c|c|c|c|}
        \hline
        Numer bita & 1 & 2 & 3 & 4 & 5 & 6 & 7 & \textcolor{green}{8} & \textcolor{green}{9} & \textcolor{green}{10} \\
        \hline
        Kod & 1 & 1 & 1 & 1 & 0 & 1 & 0 & \textcolor{green}{1} & \textcolor{green}{0} & \textcolor{green}{1} \\
        \hline
    \end{tabular}
\end{center}
Otrzymujemy na końcu taki kod:
\[
    k[n]=[1,1,1,1,0,1,0,1,0,1]
\]

\section{Przekształcenie Fouriera sygnału}
Przykładowe zadanie:
\[
    \text{Sygnał: }\cos (4t)
\]
\[
    \text{Fala nośna: }2 \cos(20t)
\]
\end{document}