\documentclass[11pt]{article}
\usepackage[utf8]{inputenc}
\usepackage[T1]{fontenc}
\usepackage{amsmath, amssymb, amsthm}
\usepackage[hidelinks]{hyperref}
\usepackage{listings}

\title{Wbudy do budy}
\author{Dawid Gradowski (puckmoment na dc)}
\date{Luty 2025}
\begin{document}
\maketitle
\renewcommand\contentsname{Spis treści}
\tableofcontents
\newpage
\section{Prolog}
Notatki robione w oparciu o projekt, który robiłem sam na zajęciach (Licznik jak coś).

\begin{center}
    \begin{tabular}{ |c|c|c| }
        \hline
        \multicolumn{3}{|c|}{Komponenciki}\\
        \hline
        Płytka & LPC1768/9 & Intrukcja \\ 
        \hline
        Ekran OLED & Rodzaj & Intrukcja \\
        \hline
        Termometr & LM75 & Instrukcja \\
        \hline  
    \end{tabular}
\end{center}
\section{GPIO}
\subsection{Guziczki}

\section{SPI}
\subsection{Wyświetlacz}
\subsection{Zapis na kartę pamięci}

\section{I$^2$C}
\subsection{Czujnik natężenia światła}
\subsection{Termometr}
W przypadku termomemetru LM75 adres jest ustalany następująco:
\begin{center}
    \begin{tabular}{|c|c|c|c|c|c|c|}
        \hline
         & & & & $A_2$ & $A_1$ & $A_0$\\
        \hline
        1& 0& 0& 1& X & X & X\\
        \hline
    \end{tabular}
\end{center}
Pierwsze 4 bity są odczytane z instrukcji. Bity oznaczone $A_x$ są ustalane
zależnie od termometra na podstawie lutowania. Jeśli $A_x$ jest przylutowany
do gruntu (ground) to w adresie mamy 0, a jeśli do $+V_S$ to 1.
\section{RTC}
\end{document}