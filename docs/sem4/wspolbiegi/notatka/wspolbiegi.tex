\documentclass[11pt]{article}
\usepackage{float}
\usepackage[utf8]{inputenc}
\usepackage[T1]{fontenc}
\usepackage{amsmath, amssymb, amsthm}
\usepackage[hidelinks]{hyperref}
\usepackage{listings}
\usepackage{xcolor}
\usepackage{graphicx}
\usepackage[UTF8]{ctex}

\renewcommand{\figurename}{Rysunek}
\renewcommand{\tablename}{Tabela}


\title{Egzamin - współbiegi dla kolegi}
\author{Dawid Gradowski (puckmoment na dc)}
\date{11 czerwca 2025}
\begin{document}
\maketitle
\renewcommand\contentsname{Spis treści}
\tableofcontents
\newpage
\section{Czasowe miary efektywności zrównoleglenia.}
\section{Sprawność i skalowalność.}
\section{Klasyfikacja maszyn równoległych (Taksdnomia Flynna).}
\section{Klasy komputerów równoległych ze względu na dostęp do pamięci operacyjnej.}
\section{Wyidealizowany model PRAM (opisać, podać klasyfikację).}
\section{Opisać przełączenie kontekstu pomiędzy procesami.}
\section{Atomowość instrukcji (podać ogólną definicję, wyjaśnić pojęcie atomowości na poziomie}
\section{zętu, i języków programowania), przeplot danych (omówić zagadnienie, podać przykłady).}
\section{Omówić działanie procesów i wątków dla maszyn z pamięcią wspólną.}
\section{Komunikacja pomiędzy wątkami, zmienne lokalne dla maszyn z pamięcią wspólną.}
\section{Komunikacja pomiędzy procesami, zmienne wspólne dla maszyn z pamięcią wspólną.}
\section{ Omówić działanie zamka (zamek, zamek czytelnicy pisarze).}
\section{Omówić działanie semafora.}
\section{Omówić działanie bariery.}
\section{Omówić działanie monitora.}
\section{Omówić działanie mechanizmu wirtualnej pamięci wspólnej.}
\section{Omówić rodzaje komunikacji w równoległych systemach rozproszonych.}
\section{Omówić podstawowe modele programów równoległych.}
\section{Omówić rodzaje komunikacji ze względu na sterowanie nią.}
\section{Omówić rodzaje komunikacji dla maszyn z pamięcią lokalną.}
\end{document}