\documentclass[11pt]{article}
\usepackage[utf8]{inputenc}
\usepackage[T1]{fontenc}
\usepackage{amsmath, amssymb, amsthm}
\usepackage[hidelinks]{hyperref}

\title{Analiza Matematyczna 2 - poradnik}
\author{Dawid Gradowski (puckmoment na dc)}
\date{Luty 2025}
\begin{document}
\maketitle
\renewcommand\contentsname{Spis treści}
\tableofcontents
\newpage
\section{Prolog}
\vbox{
    Ten dokument tworzę dlatego by poznać wspaniałe narzędzie
    jakim jest {\LaTeX} oraz by przygotować się do egzaminu
    z Analizy Matematycznej 2, ale to nie są jedyne powody.
    Tworzę ją jako człowiek z misją, człowiek który upierdolił wszystkie
    3 kartkówki, z czego na 2 z nich dostał 0 punktów. Nie lękajcie się jednak
    czytelnicy tej notatki, ponieważ jestem dowodem na to, że nawet z najgorszych
    sytuacji da się wyjść obronną ręką.
}

\section{Przydatne rzeczy do wszystkiego}
\subsection{Szeregi harmoniczne różnej krotności}
\vbox{
    Jednym z typowych szeregów jest 
    tak zwany \textbf{szereg} \textbf{harmoniczny}.
    Znajomość tego szeregu oraz kiedy jest on zbieżny
    jest kluczowa jeśli chce się zdać ten przedmiot. Szereg ten wygląda
    następująco:
    \[\sum_{n=1}^{\infty}\frac{1}{n^p}\]
    gdzie:
    \begin{list}{$\bullet$}{}
        \item  dla $p \leq 1$ ciąg ten jest rozbieżny
        \item  dla $p > 1$ ciąg ten jest zbieżny
    \end{list}
    Szeregi harmoniczne oraz ich zbieżności bardzo nam się przydadzą
    w momencie gdy będziemy korzystać w kryterium porównawczego
    o którym można przeczytać w dalszej części notatki.
}
\subsection{Wzór na $e$}
\vbox{
    Bardzo ważnym wzorem w zadaniu pierwszym na kartkówce pierwszej
    jest coś co ja nazywam wzorem na $e$. Wygląda on następująco:
    \[\lim_{n\to\infty}(1 + \frac{1}{n})^n = e\]
    Wzoru tego trzeba było użyć na każdej kartkówce nr 1 jaką rozwiązywałem
    oraz zawsze w tym samym miejscu, w zadaniu 1
    w przykładzie w którym używało się kryterium Cauchy'ego. Nigdy nie będzie on
    tak ładnie widoczny jak powyżej więc na przykładzie zadania z kartkóweczki
    wytłumacze jak uzyskać odpowiednią postać. Na kartkówce był taki 
    przykład:
    \[
        \sum_{n=1}^{\infty}(\frac{3n^2 - n}{3n^2 + 4n})^{n^2}
    \]
    W trakcie stosowania kryterium Cauchy'ego otrzymujemy następującą postać:
    \[\lim_{n\to\infty}(\frac{3n^2 - n}{3n^2 + 4n})^n\]
    W pierwszej kolejności możemy wyciągnąć $n$ przed nawias by w nawiasie zostały
    nam $n$ pierwszej potęgi.
    \[\lim_{n\to\infty}(\frac{3n^2 - n}{3n^2 + 4n})^n = 
        \lim_{n\to\infty}(\frac{n(3n - 1)}{n(3n + 4)})^n =
        \lim_{n\to\infty}(\frac{3n - 1}{3n + 4})^n
    \]
    Aby w nawiasie uzyskać postać $1 + \frac{1}{x}$
    trzeba uzyskać w liczniku $3n + 4$ tak samo jak jest w mianowniku.
    $3n - 1 = 3n + 4 - 5$ więc można to zapisać następująco:
    \[\lim_{n\to\infty}(\frac{3n - 1}{3n + 4})^n = 
        \lim_{n\to\infty}(\frac{3n + 4 - 5}{3n + 4})^n =
        \lim_{n\to\infty}(\frac{3n + 4}{3n + 4} + \frac{-5}{3n + 4})^n = 
        \lim_{n\to\infty}(1 + \frac{-5}{3n + 4})^n
    \]
    Już coraz bardziej wygląda to jak postać której szukamy. Brakuje nam jedyneczki w liczniku.
    Aby mieć jedynkę w liczniku trzeba górę i dół przemnożyć przez $-\frac{1}{5}$.
    \[
        \lim_{n\to\infty}(1 + \frac{-5}{3n + 4})^n = 
        \lim_{n\to\infty}(1 + \frac{-5}{3n + 4} \times \frac{-\frac{1}{5}}{-\frac{1}{5}})^n\ = 
        \lim_{n\to\infty}(1 + \frac{1}{-\frac{3n}{5} - \frac{4}{5}})^n
    \]
}
\vbox{
    No i teraz mamy postać typu kongo, która wygląda dodupnie, ale spokojnie bo to
    już ostatnia prosta. Wytłumaczę na prostym przykładzie co trzeba zrobić:
    \[
        \lim_{n\to\infty}(1 + \frac{1}{x})^y =
        \lim_{n\to\infty}((1 + \frac{1}{x})^x)^{\frac{y}{x}} = 
        e^{\lim_{n\to\infty}\frac{y}{x}}
    \]
    Robiąc to samo na naszym przykładzie otrzymamy:
    \[
        \lim_{n\to\infty}(1 + \frac{-5}{3n + 4})^n =
        \lim_{n\to\infty}((1 + \frac{1}{-\frac{3n}{5} - \frac{4}{5}})^{-\frac{3n}{5} - \frac{4}{5}})^{\frac{n}{-\frac{3n}{5} - \frac{4}{5}}} = 
        e^{-\frac{5}{3}} = \frac{1}{e^{\frac{5}{3}}}
    \]
    I tak właśnie z jakiegoś gówna dostaliśmy piękne $e$.
}
\subsection{Reguła łańcuchowa}
\vbox{
    Reguła łańcuchowa to jedno z podstawowych narzędzi rachunku różniczkowego
    (liczenia pochodnych), 
    które pozwala na liczenie pochodnych funkcji złożonych. Reguła wygląda następująco.
    Zakładamy, że mamy funkcję składającą się z 2 funkcji:
    \[
        h(x) = g(f(x))
    \]
    gdzie:
    \begin{list}{$\bullet$}{}
        \item $f(x)$ to funkcja wewnętrzna
        \item $g(u)$ to funkcja zewnętrzna, przy czym nasze $u = f(x)$
    \end{list}
    Aby znaleźć pochodną $h(x)$, stosujemy regułę łańcuchową:
    \[
        h'(x) = g'(f(x)) \cdot f'(x)
    \]
    Jak stosować to w praktyce. Weźmy sobie taki przykład:
    \[
        h(x)=\sqrt{5x + 3} = (5x + 3)^{\frac{1}{2}}
    \]
    I teraz określamy sobie co jest funkcję wewnętrzną, a co zewnętrzną:
    \begin{list}{$\bullet$}{}
        \item funkcja wewnętrzna: $f(x) = 5x + 3$
        \item funkcja zewnętrzna: $g(u) = u^{\frac{1}{2}}$, przy czym nasze $u = f(x)$
    \end{list}
    Licząc pochodnę otrzymujemy:
    \begin{list}{$\bullet$}{}
        \item $f'(x) = 5$
        \item $g'(u) = \frac{1}{2} \cdot u^{\frac{1}{2} - 1} = \frac{1}{2} \cdot u^{-\frac{1}{2}}
        = \frac{1}{2} \cdot (5x + 3)^{-\frac{1}{2}} = \frac{1}{2\sqrt{5x + 3}}$
    \end{list}
    Wstawiając więc otrzymane pochodne do wzoru na regułę łańcuchową otrzymujemy:
    \[
        h'(x) = \frac{1}{2\sqrt{5x + 3}} \cdot 5
        = \frac{5}{2\sqrt{5x + 3}}
    \]

}
\subsection{Przydatne granice}
\vbox{
    \[
        \lim_{x \to 0} \frac{\arcsin x}{x} = 1
    \]
    \[
        \lim_{x \to 0} \frac{\sin x}{x} = 1
    \]
    \[
        \lim_{x \to 0} \frac{\text{arctg\space}x}{x} = 1
    \]
    \[
        \lim_{x \to 0} \frac{\text{tg\space}x}{x} = 1
    \]
    \[
        \lim_{x \to 0} \frac{\ln(1 + x)}{x} = 1
    \]
    \[
        \lim_{x \to 0} \ln(x) = -\infty
    \]
    \[
        \lim_{x \to 0} \frac{e^x - 1}{x} = 1
    \]
    \[
        \lim_{x \to 0} \frac{1 - \cos x}{x} = \frac{1}{2}
    \]
    \[
        \lim_{x \to 0} \frac{\log_{a}(1 + x)}{x} = log_{a}e
    \]
    \[
        \lim_{x \to 0} \frac{a^x - 1}{x} = \ln a
    \]
    \[
        \lim_{x \to \infty} (1 + \frac{a}{x})^x = e^a
    \]
    \[
        \lim_{x \to \infty} \ln(x) = \infty
    \]
}
\subsection{Przydatne wzory na pochodne}
\vbox{
    \[
        f(x)=x^n \Rightarrow f'(x)=n \cdot x^{n-1}
    \]
    Ciekawe przykłady zastosowania:
    \[
        f(x)=\frac{1}{x}=x^{-1} \Rightarrow f'(x) = (-1) \cdot x^{-2} = -\frac{1}{x^2}
    \]
    \[
        f(x)= \sqrt{x} = x^{\frac{1}{2}} \Rightarrow f'(x) = \frac{1}{2} \cdot x^{\frac{1}{2} - 1} 
        = \frac{1}{2 \cdot \sqrt{x}}
    \]
    Te mniej typowe wzory:
    \[
        f(x)=a^x \Rightarrow f'(x)=a^x \cdot \ln a
    \]
    \[
        f(x)=e^x \Rightarrow f'(x)=e^x
    \]
    \[
        f(x)=\log_{a}x \Rightarrow f'(x)= \frac{log_{a}e}{x}
    \]
    \[
        f(x)=\ln x \Rightarrow f'(x)=\frac{1}{x}
    \]
    \[
        f(x)=\sin x \Rightarrow f'(x)=\cos x
    \]
    \[
        f(x)=\cos x \Rightarrow f'(x)=-\sin x
    \]
    \[
        f(x)=\text{tg\space}x \Rightarrow f'(x)=\frac{1}{\cos^2 x}
    \]
    \[
        f(x)=\text{ctg\space}x \Rightarrow f'(x)=-\frac{1}{\sin^2 x}
    \]
    \[
        f(x)=\arcsin x \Rightarrow f'(x)=\frac{1}{\sqrt{1-x^2}}
    \]
    \[
        f(x)=\arccos x \Rightarrow f'(x)=-\frac{1}{\sqrt{1-x^2}}
    \]
    \[
        f(x)=\text{arctg\space} x \Rightarrow f'(x)=\frac{1}{1+x^2}
    \]
    \[
        f(x)=\text{arcctg\space} x \Rightarrow f'(x)=-\frac{1}{1+x^2}
    \]
}

\subsection{Inne przydatne własności}
\vbox{
    \[
        \text{tg}x = \frac{\sin x}{\cos x}
    \]
    \[
        \text{ctg}x = \frac{\cos x}{\sin x}
    \]
}

\section{Kryteria określające zbieżność}
\vbox{
    Kryteria opisane poniżej pozwalają określić czy dany szereg
    jest zbieżny czy rozbieżny. Generalnie to chuj wie co to znaczy,
    jak chcesz wiedzieć to sobie w google wpisz, ale to nie jest wiedza
    potrzebna do zdania.
}

\subsection{Kryterium Cauchy'ego}
\vbox{
    Dla szeregu $\sum_{n = 1}^{\infty}{a_n}$ jego zbieżność wyznaczamy
    licząc $g$ albo $q$ (nie mogłem się odczytać w notatkach jaka to literka
    ale ja na kartkówce co zdałem pisałem $g$).
    \[
        g = \lim_{n\to\infty}{\sqrt[n]{\left\lvert a_n \right\rvert}}
    \]
    Najczęściej wartość bezględna $\left\lvert a_n \right\rvert$ to po prostu $a_n$. Posiadając policzone 
    $g$ dowiadujemy się, że:
    \begin{list}{$\bullet$}{}
        \item $g < 1$ to szereg jest bezwzględnie zbieżny
        \item $g > 1$ to szereg jest rozbieżny
        \item $g = 1$ to kryterium jest niewystarczające, więc nie rozstrzyga zbieżności
    \end{list}
    Kryterium to najlepiej stosować gdy całe wyrażenie jest do potęgi $n^p$, gdzie
    $p$ to jakaś liczba całkowita.
}

\subsection{Kryterium d'Alemberta}
\vbox{
    Tak samo jak w poprzednik kryterium, dla szeregu $\sum_{n = 1}^{\infty}{a_n}$ jego zbieżność wyznaczamy
    licząc $g$. Zmienia się jednak wzór
    \[
        g = \lim_{n\to\infty}{\left\lvert \frac{a_{n + 1}}{a_n} \right\rvert}
    \]
    Najczęściej wartość bezględna $\left\lvert \frac{a_{n + 1}}{a_n} \right\rvert$ to po prostu $\frac{a_{n + 1}}{a_n}$. Posiadając policzone 
    $g$ tak samo jak w poprzednim dowiadujemy się, że:
    \begin{list}{$\bullet$}{}
        \item $g < 1$ to szereg jest bezwzględnie zbieżny
        \item $g > 1$ to szereg jest rozbieżny
        \item $g = 1$ to kryterium jest niewystarczające, więc nie rozstrzyga zbieżności
    \end{list}
    Kryterium to najlepiej stosować gdy w szeregu występują jakieś silnie albo $n$-te potęgi.
}

\subsection{Kryterium porównawcze}
\vbox{
    Jeśli mamy 2 szeregi dodatnie
    \[
        (A) \sum_{n=1}^{\infty}a_n
    \]
    \[
        (B) \sum_{n=1}^{\infty}b_n
    \]
    i dla prawie wszystkich $n$ zachodzi $a_n \leq b_n$ to:
    \begin{list}{$\bullet$}{}
        \item ze zbieżności szeregu $(B)$ wynika zbieżność szeregu $(A)$
        \item z rozbieżności szeregu $(A)$ wynika zbieżność szeregu $(B)$
    \end{list}
    Ewentualnie jak komuś łatwiej zapamiętać to:
    \begin{list}{$\bullet$}{}
        \item ze zbieżności większego szeregu możemy udowodnić zbieżność szeregu mniejszego
        \item z rozbieżności szeregu mniejszego możemy udowodnić rozbieżność szeregu większego
    \end{list}
    Prawie wszystkie $n$ oznacza, że dla jakiegoś $n$ staje się to prawdą, bo jak któreś $n$ jest prawdą
    to później idzie to w nieskończoność, a to całkiem sporo przypadków jakby nie patrzeć.
    Zwykle jest tak, że już dla $n = 1$ jest to prawdą, więc nie ma się czym martwić. \\
    Kryterium to najlepiej stosować gdy dla dużych $n$ nasze $\sum_{n=1}^{\infty}a_n$ przypomina szereg
    harmoniczny jakiejkolwiek krotności. Tylko co to znaczy, że dla dużych $n$ coś przypomina nam coś?
    To bardzo proste. Weźmy dla przykładu taki szereg:
    \[
        \sum_{n = 1}^{\infty} \frac{n + 1}{3n^2 - n}
    \]
    Dla małych $n$ jedynka w liczniku ma dość duży wpływ na wartość wyrażenia, 
    ale gdyby nasze $n$ było olbrzymie to byłby to że tak to ujmę jeden chuj, dlatego:
    \[
        \sum_{n = 1}^{\infty} \frac{n + 1}{3n^2 - n} \approx 
        \sum_{n = 1}^{\infty} \frac{n}{3n^2 - n}
    \]
}
\vbox{
    Mając taką postać możemy wyciągnąć $n$ przed nawias
    \[
        \sum_{n = 1}^{\infty} \frac{n}{3n^2 - n} = 
        \sum_{n = 1}^{\infty} \frac{n}{n(3n - 1)} = 
        \sum_{n = 1}^{\infty} \frac{1}{3n - 1} \approx 
        \sum_{n = 1}^{\infty} \frac{1}{3n} = 
        \frac{1}{3} \times \sum_{n = 1}^{\infty} \frac{1}{n}
    \]
    Możemy teraz zauważyć podobieńtwo do szeregu harmonicznego o którym pisałem w 
    części 2.1. Jest to szereg harmoniczny pierwszego stopnia, który jest rozbieżny.
    Teraz aby udowodnić rozbieżność znajdźmy szereg harmoniczny mniejszy od naszego ciągu.
    \[
        \sum_{n = 1}^{\infty} \frac{n + 1}{3n^2 - n} \geq 
        \sum_{n = 1}^{\infty} \frac{1}{4n}
    \]
    Na mocy kryterium porównawczego, szereg ten jest rozbieżny.
}

\subsection{Kryterium Leibniza}
\vbox{
    Jest to kryterium zbieżności szeregów naprzemiennych, które
    określa warunkową zbieżność szeregu. Nie do końca wiem co to znaczy ale
    tak jest. Szereg $\sum_{n=0}^{\infty}(-1)^n a_n$ jest zbieżny warunkowo gdy:
    \begin{enumerate}
        \item $\lim_{n \to \infty} a_n = 0$,
        \item Ciąg $a_n$ jest nierosnący, czyli $a_n - a_{n + 1} \geq 0$
    \end{enumerate}
    \textbf{UWAGA! Trzeba pamiętać, że określa to wyłącznie zbieżność warunkową
    szeregu a nie bezwzględną!}
}

\section{Szeregi potęgowe}
\vbox{
    Wzór ogólny szeregu potęgowego wygląda następująco:
    \[
        \sum_{n = 0}^{\infty} a_n (x - x_0)^n
    \]
}
\vbox{
    Dla takiego szeregu potęgowego możemy wyliczyć promień zbieżności $r$.
    Promień wykorzystujemy przy liczeniu przedziału zbieżności.
}
\vbox{
    Przedział zbieżności określa dla jakich wartości $x$ szereg ten jest 
    zbieżny, nie określa on jednak zbieżności na granicach przedziału.
    Szereg jest zbieżny gdy:
    \[
        x \in (x_0 - r ; x_0 + r)
    \]
}
\vbox{
    Obszar zbieżności jest tym samym co przedział zbieżności ale z uwzględnieniem
    zbieżności dla wartości skrajnych.
}
\subsection{Kryterium Cauchy'ego - Hadamarda}
\vbox{
    Kryterium to określa promień zbieżności szeregu potęgowego.
    Jest ono bardzo podobne do kryterium Cauchy'ego.
    Zaczynamy od policzenia $g$ z tego samego wzoru,
    którego używaliśmy przy kryterium Cauchy'ego
    \[
        g = \lim_{n\to\infty}{\sqrt[n]{\left\lvert a_n \right\rvert}}
    \]
    Teraz w zależności od obliczonego $g$, nasze $r$ jest równe:
    \begin{equation}
        r=\begin{cases}
          \infty, & {\text{if }} g = 0\\
          0, & {\text{if }} g = \infty\\
          \frac{1}{g}, & {\text{if }} g < 0 < \infty
        \end{cases}
    \end{equation}
}

\subsection{Kryterium d'Alemberta (szeregi potęgowe)}
\vbox{
    Tak jak wykorzystaliśmy kryterium Cauchy'ego tak bliźniaczo możemy wykorzystać
    kryterium d'Alemberta.
    \[
        g = \lim_{n\to\infty}{\left\lvert \frac{a_{n + 1}}{a_n} \right\rvert}
    \]
    i następnie sytuacja wygląda dokładnie tak samo jak w poprzednim kryterium, więc
    używamy równania (1). Można też zauważyć, że nasze $r$, jeśli granica istnieje
    jest po prostu równe 
    \[
        r = \lim_{n\to\infty}{\left\lvert \frac{a_{n}}{a_{n + 1}} \right\rvert}
    \]
}

\section{Rozwijanie funkcji w szereg}
\vbox{
    Istnieje coś takiego jak szereg Taylora, który wygląda następująco:
    \[
        \sum_{n = 0}^{\infty}\frac{f^{(n)}(x_0)}{n!}(x - x_0)
    \]
    Na kartkówcę drugiej trzeba będzie najprawdopodobniej rozwinąć funkcję w szeregu
    Maclaurina. Jest to szereg Taylora dla którego $x_0 = 0$.
    Rozwinięcie funkcji w szereg polega na wyliczeniu dużej ilości pochodnych tak,
    aby w pewnym momencie wyznaczyć wzór na $n$-tą pochodną. Jak pierwszy raz
    zobaczyłem, że muszę coś takiego zrobić to się załamałem i dałem ff, ale ty tak
    nie rób. Pokażę ci zaraz wspaniałe wzorki :).
}

\subsection{Funkcja z kreską ułamkową}
\vbox{
    Nie będę zgrywał bohatera i po prostu się przyznam, że nie wiedziałem
    jak nazwać tą fukcję, ale ta funkcja wygląda podobnie do czegoś takiego:
    \[
        f(x) = \frac{3x}{8 + x^3}
    \]
    Jeśli mamy coś takiego w zadaniu to należy zapamiętać, że:
    \[
        \frac{1}{1 - q} = \sum_{n = 0}^{\infty}q^n
        \text{\space lub \space}
        \frac{1}{1 + q} = \sum_{n = 0}^{\infty}(-1)^n q^n
    \]
    wtedy kiedy $q \in (-1, 1)$. Dla przykładu rozwińmy sobie podaną wyżej funkcje. W liczniku potrzebujemy samej
    $1$, a w mianowniku $1 + q$ z czego $q$ może być czymś dziwnym, a więc
    \[
        \frac{3x}{8 + x^3} = 
        3x \times \frac{1}{{8 + x^3}} =
        \frac{3x}{8} \times \frac{1}{{1 + \frac{x^3}{8}}} = 
        \frac{3x}{8} \times \sum_{n = 0}^{\infty}(-1)^n (\frac{x^3}{8})^n
    \]
    Mamy już szereg ale to nie koniec zadania. Teraz musimy pomnożyć
    to co jest przed szeregiem z tym co jest w szeregu. Nie ma się tu czego
    bać po prostu wymnażamy jakby tego $\sum$ nie było. Można sam szereg sobie
    ładniej zapisać:
    \[
        \frac{3x}{8} \times \sum_{n = 0}^{\infty}(-1)^n (\frac{x^3}{8})^n = 
        \frac{3x}{8} \times \sum_{n = 0}^{\infty}(-1)^n \frac{x^{3n}}{8^{n}} = 
        \sum_{n = 0}^{\infty}(-1)^n \frac{3}{8^{n + 1}}x^{3n + 1}
    \]
    Trzeba też wyznaczyć dla jakich $x$ stwierdzenie $\frac{x^3}{8} \in (-1, 1)$ co jest równoznawcze z
    \[
        -1 < \frac{x^3}{8} < 1
    \]
    \[
        -8 < {x^3} < 8
    \]
    \[
        -2 < x < 2
    \]
}

\subsection{Funkcja z logarytmen naturalnym}
\vbox{
    Wzór wygląda następująco
    \[
        ln(1 + t) = \sum_{n = 0}^{\infty} \frac{(-1)^{n}}{n + 1}t^{n+1}
    \]
    gdzie $t \in (-1, 1)$. 
    Zasada używania wzoru jest podobna jak w sekcji 5.1. Przykład:
    \[
        f(x) = x \ln(1 - 2x) = 
        x \sum_{n = 0}^{\infty} \frac{(-1)^{n}}{n + 1}(-2x)^{n+1}
    \]
    \[
        = \sum_{n = 0}^{\infty}(-1)^{n + 1} \cdot (-1)^{n} \cdot \frac{2^{n + 1}}{n + 1} x^{n + 2}
        = \sum_{n = 0}^{\infty} (-1) \cdot \frac{2^{n + 1}}{n + 1} x^{n + 2}
    \]
}

\subsection{Inne funkcje}
\vbox{
    Istnieją jeszcze 3 wzory, ale nie są one potrzebne do zdania kartkówek.
    Te wzory to:
    \[
        e^x = \sum_{n = 0}^{\infty} \frac{x^n}{n!} \text{\space dla \space} x \in R
    \]
    \[
        \sin{x} = \sum_{n = 0}^{\infty} (-1)^n \cdot \frac{x^{2n + 1}}{(2n + 1)!} \text{\space dla \space} x \in R
    \]
    \[
        \cos{x} = \sum_{n = 0}^{\infty} (-1)^n \cdot \frac{x^{2n}}{(2n)!} \text{\space dla \space} x \in R
    \]
}

\section{Funkcje 2 zmiennych}
\subsection{Znajdowanie granicy (sposób z t)}
\vbox{
    Znajdziemy sobie granicę takiej dzikiej funkcji 2 zmiennych:
    \[
        \lim_{(x,y) \to (0,0)}(x^2 + y^2) \text{ctg} (3x^2 + 3y^2)
    \]
    Możemy zauważyć, że po lekkim przekształceniu nasze zmienne $x$ i $y$ występują
    tylko w postaci $x^2 + y^2$:
    \[
        \lim_{(x,y) \to (0,0)}(x^2 + y^2) \text{ctg} (3x^2 + 3y^2) = 
        \lim_{(x,y) \to (0,0)}(x^2 + y^2) \text{ctg} (3 \cdot (x^2 + y^2))
    \]
    W takiej sytuacji możemy za $x^2 + y^2$ podstawić $t$ i nie będzie to miało wpływu
    na granice gdy $t$ będzie dążyć do $0$:
    \[ 
        \lim_{(x,y) \to (0,0)}(x^2 + y^2) \text{ctg} (3 \cdot (x^2 + y^2))
        = \lim_{t \to 0} t \cdot \text{ctg}(3t)
    \]
    I to jest moment w którym trzeba coś wiedzieć. Do tych zadań trzeba dobrze znać 
    regułę łańcuchową opisaną w rozdziale 2.3 oraz granice opisane w rozdziale 2.4.
    Nie ma więc jednego sprawdzonego sposobu, trzeba po prostu się troche obyć w tych zadaniach albo 
    olać i liczyć na to, że punkty z reszty zadań wystarczą by zdać. Granica naszego przykładu jest równa:
    \[ 
        \lim_{t \to 0} t \cdot \text{ctg}(3t) =
        \lim_{t \to 0} t \cdot \frac{\cos 3t}{\sin 3t} =
        \lim_{t \to 0} \frac{1}{3} \cdot (\frac{\sin 3t}{3t})^{-1} \cdot \cos 3t = 
        \frac{1}{3} \cdot 1 \cdot 1^{-1} = \frac{1}{3}
    \]
    To co jest napisane powyżej jest bardzo sprytne ale teraz pozostaje pytanie: czy to jest 
    coś na co bym wpadł na kartkówce? Nie wydaje mi się, ale punkty z pozostałych zadań pozwalają zdać
    więc na giga luzie.
}
\subsection{Pokazywanie, że granica nie istnieje}
\vbox{
    Weźmy sobie taki przykład:
    \[
        \lim_{(x,y) \to (0,0)}\frac{x^3y}{3x^4 + 2y^4}
    \]
    Musimy znaleźć takie 2 ciągi, które również będą dążyły do $(0,0)$. Jeśli nie rozumiesz
    to spójrz na takie dwa przykładowe ciągi, których ty też możesz użyć dla dowolnego przypadku:
    \[
        (\frac{1}{n},\frac{1}{n}) \stackrel{n \to \infty}{\to} (0, 0)
    \]
    \[
        (\frac{2}{n},\frac{1}{n}) \stackrel{n \to \infty}{\to} (0, 0)
    \]
    Możemy zauważyć, że tak na prawdę możemy użyć $\frac{x}{n}$ gdzie $x$ jest dowolną liczbą,
    a $n$ dąży do nieskończoności. Gdy już wybraliśmy sobie ciągi sprawa jest giga
    prosta. Podstawiamy sobie nasze ciągi do naszej funkcji:
    \begin{list}{$\bullet$}{}
        \item Dla $(\frac{1}{n},\frac{1}{n})$
            \[
                \lim_{(x,y) \to (0,0)}\frac{x^3y}{3x^4 + 2y^4} = 
                \lim_{n \to \infty} \frac{\frac{1}{n^3} \cdot \frac{1}{n}}{\frac{3}{n^4} + \frac{2}{n^4}} =
                \lim_{n \to \infty} \frac{\frac{1}{n^4}}{\frac{5}{n^4}}
                = \frac{1}{5}
            \]
        \item Dla $(\frac{2}{n},\frac{1}{n})$
            \[
                \lim_{(x,y) \to (0,0)}\frac{x^3y}{3x^4 + 2y^4} =
                \lim_{n \to \infty} \frac{\frac{8}{n^3} \cdot \frac{1}{n}}{\frac{3 \cdot 16}{n^4} + \frac{2}{n^4}} =
                \frac{8}{50} = \frac{4}{25}
            \]
    \end{list}
    Otrzymaliśmy 2 różne granice dlatego wiemy, że granica nie istnieje. 
    Otrzymanie różnych granic w taki sposób udowadnia nam, że granica nie istnieje
    \textbf{ ale otrzymanie w taki sposób takich samych granic nie gwarantuje, że granica istnieje.}
    W nieskończonej ilości kombinacji ciągów, mogą się znaleźć 2 takie dla których wyjdą różne granice.
}


\subsection{Obliczanie wartości przybliżonej}
\vbox{
    Obliczmy wartość przybliżoną funkcji na przykładzie:
    \[
        f(x, y) = \sqrt{5x - y^2}
    \]
    \[
        P = (1.04;-0.97) = (x, y)
    \]
    Przyjmujemy sobie punkt blisko naszego punku $P$:
    \[
        P_0 = (1; -1) = (x_0, y_0)
    \]
    Jeśli funkcja $f$ jest różniczkowalna w punkcie $(x_0,y_0)$ to możemy 
    podać przybliżone wartości "blisko" punktu $(x_0, y_0)$. Jeśli nie rozumiecie
    o co chodzi to pełen chill, zaraz wszystko się wyklaruje. Istnieje taki wzór:
    \[
        f(x, y) \simeq f(x_0, y_0) + f'x(x_0, y_0) \cdot \Delta x + f'y(x_0, y_0) \Delta y
    \]
    Wartości $\Delta x$ oraz $\Delta y$ to różnice współrzednych punktów $P$ i $P_0$, czyli:
    \[
        \Delta x = x - x_0 = 1.04 - 1 = 0.4 = \frac{4}{100}
    \]
    \[
        \Delta y = y - y_0 = -0.97 - (-1) = 0.3 = \frac{3}{100}
    \]
    Aby wyznaczyć $f'x(x_0, y_0)$ oraz $f'y(x_0, y_0)$ trzeba użyć reguły łańcuchowej opisanej
    w rozdziale 2.3. Otrzymamy z tego:
    \[
        f'x(1, -1) = \frac{1}{2\sqrt{5x - y^2}} \cdot 5 = \frac{5}{2\sqrt{5 - 1}} = \frac{5}{4}
    \]
    \[
        f'y(1, -1) = \frac{1}{2\sqrt{5x - y^2}} \cdot (-2y) = \frac{2}{2\sqrt{5 - 1}} = \frac{1}{2}
    \]
    Z kolei nasze $f(x_0, y_0)$ jest równe:
    \[
        f(1, -1) = \sqrt{5x - y^2} = \sqrt{5 - 1} = 2
    \]
    Podstawiając wszystkie wartości do wzoru na przybliżenie $f(x,y)$ otrzymujemy:
    \[
        f(x,y) \eqsim 2 + \frac{5}{4} \cdot \frac{4}{100} + \frac{1}{2} \cdot \frac{3}{100}
        = 2 + \frac{10}{200} + \frac{3}{200} = 2 + \frac{13}{200} = 2.065
    \]

}
\subsection{Wyznaczanie ekstremów}
\vbox{
    Wyznaczmy ekstrema lokalne poniższej funkcji.
    \[
        f(x, y) = x^2 - x^2y^3 + 3y^2 - 3y + 1
    \]
    Pierwszym krokiem w liczeniu ekstremów jest wyznaczenie dziedziny. Jak to zrobić
    każdy mniej więcej powinien wiedzieć, dla każdej funkcji robi się to inaczej.
    W przypadku naszej funkcji dziedzina to:
    \[
        (x, y) \in R^2
    \]
}
\vbox{
    Kolejnym krokiem jest sprawdzenie dla jakiego punktu/punktów zachodzi warunek konieczny.
    Warunkiem koniecznym tego czy dany punkt, np. $(x_0, y_0)$ może być ekstremum lokalnym jest, że dla tego punktu
    pochodna $x$-owa oraz pochodna $y$-owa muszą być równe 0.
    \[
        f'x(x_0, y_0) = 0
    \]
    \[
        f'y(x_0, y_0) = 0
    \]
    W przypadku naszej funkcji te pochodne to:
    \[
        f'x(x, y) = 2x - 2xy^3
    \]
    \[
        f'y(x, y) = -3x^2y^2 + 6y - 3
    \]
    Tworzymy układ równań z którego wynika:
    \[
        \begin{cases}
            2x - 2xy^3 = 0 /:2\\
            -3x^2y^2 + 6y - 3 = 0 /:3\\
        \end{cases}
        \Rightarrow
        \begin{cases}
            x - xy^3 = 0 \Rightarrow x (1 - y^3) = 0\\
            -x^2y^2 + 2y - 1 = 0\\
        \end{cases}
    \]
    Z pierwszego równania otrzymujemy:
    \[
        x = 0 \lor 1 - y^3 = 0 \to y^3 = 1 \to y = 1
    \]
    Podstawiamy otrzymane wartości do drugiego równania:
    \begin{list}{$\bullet$}{}
        \item Dla $x = 0$: \\
                \[
                    2y - 1 = 0 \text{\space więc \space} y = \frac{1}{2}
                \]
                Otrzymujemy więc punkt $(0, \frac{1}{2})$.
        \item Dla $y = 1$: \\
        \[
            -x^2 + 2 - 1 = 0 \text{\space więc \space} x = 1 \lor x = -1
        \]
        Otrzymujemy więc punkty $(1, 1)$ i $(-1, 1)$.
    \end{list} 
    Mamy więc 3 punkty (są to tak zwane punkty krytyczne), które będziemy musieli sprawdzić: $(0, \frac{1}{2})$, $(1, 1)$ i $(-1, 1)$.
}
\vbox{
    Następnie musimy wyznacznyć 4 drugie pochodne, aby wypełnić nimi macierz Hessego,
    zwaną również Hesjanem. W naszym przypadku nasze pochodne to:
    \[
        f''xx(x,y) = 2 - 2y^3
    \]
    \[
        f''yy(x,y) = -6x^2y + 6
    \]
    \[
        f''xy(x,y) = f''yx(x,y) = -6xy^2
    \]
    Ale czym jest Hesjan. Hesjan to macierz składająca się z drugich pochodnych
    jakiejś funkcji. Macierz ta wygląda następująco:
    \[
        H(x_0, y_0) = \begin{bmatrix}
            f''xx(x,y) & f''xy(x,y) \\
            f''yx(x,y) & f''yy(x,y)
        \end{bmatrix}
    \]
    Więc w naszym przypadku macierz ta wygląda następująco:
    \[
        H(x, y) = \begin{bmatrix}
            2 - 2y^3 & -6xy^2 \\
            -6xy^2 & -6x^2y + 6
        \end{bmatrix}
    \]
    Macierz ta jest nam potrzebna do warunku wystarczającego istnienia ekstremum.
    Warunek ten pozwala na podstawie wartości wyznacznika macierzy dla konkretnego
    punktu określić, czy jest to ekstremum czy punkt siodłowy (coś co nie jest ekstremum).
    Mamy 3 przypadki:
    \begin{list}{$\bullet$}{}
        \item $\det H(x, y) > 0$ $\to$ punkt jest ekstremum
        \item $\det H(x, y) < 0$ $\to$ punkt jest punktem siodłowym
        \item $\det H(x, y) = 0$ $\to$ nie da się określić za pomocą wyznacznika
    \end{list}
    Dodatkowo gdy określimy sobie, że punkt jest ekstremum to dla tego punktu:
    \begin{list}{$\bullet$}{}
        \item $f''xx(x,y) > 0$ $\to$ punkt jest minimum lokalnym
        \item $f''xx(x,y) < 0$ $\to$ punkt jest maksimum lokalnym
    \end{list}
}
\vbox{
    Aby obliczyć wyznacznik macierzy 2x2 należy użyć wzoru:
    \[
        \det H(x, y) = \det \begin{bmatrix}
            a & b \\
            c & d
        \end{bmatrix} = 
        \begin{vmatrix}
            a & b \\
            c & d
        \end{vmatrix} =
        a \cdot d - b \cdot c
    \]
    Obliczmy sobie wyznacznik dla poszczególnych punktów.
    \begin{list}{$\bullet$}{}
        \item Dla punktu $(0, \frac{1}{2})$
                \[
                    \det H(0, \frac{1}{2}) =
                    \begin{vmatrix}
                        \frac{7}{4} & 0 \\
                        0 & 6
                    \end{vmatrix} = \frac{7}{4} \cdot 6 - 0 \cdot 0 = \frac{21}{2} > 0
                \]
                Wiemy, że w punkcie $(0, \frac{1}{2})$ jest ekstremum.
                Widzimy również, że
                \[
                    f''xx(x,y) = \frac{7}{4} > 0
                \]
                więc punkt $(0, \frac{1}{2})$ jest minimum lokalnym.
        \item Dla punktu $(1, 1)$
                \[
                    \det H(1, 1) =
                    \begin{vmatrix}
                        0 & -6 \\
                        -6 & 0
                    \end{vmatrix} = 0 \cdot 0 - (-6) \cdot (-6) = -36 < 0
                \]
                Wiemy, że w punkcie $(1, 1)$ jest punkt siodłowy.
        \item Dla punktu $(-1, 1)$
                \[
                    \det H(-1, 1) =
                    \begin{vmatrix}
                        0 & 6 \\
                        6 & 0
                    \end{vmatrix} = 0 \cdot 0 - 6 \cdot 6 = -36 < 0
                \]
                Wiemy, że w punkcie $(-1, 1)$ jest punkt siodłowy.
    \end{list}
}
\section{Kartkówka nr 1}
Kartkówka pierwsza składa się z 2 zadań. Pierwsze zadanie polega na zbadaniu zbieżności szeregu, stosując
kryteria Cauchy'ego oraz d'Alemberta.

Drugie zadanie polega na zbadadaniu czy dany szereg jest zbieżny warunkowo czy bezwzględnie.

Poniżej zamieszczę wszystkie kartkówki jakie przerobiłem ucząć się do poprawy kartkówki, ale tylko jedno wytłumaczę dokładnie.
Gdyby ktoś chciał rozwiązania proszę do mnie napisać to mogę podesłać.

\subsection{Kartkówka wraz z dokładnym omówieniem}
\subsubsection{Zadanie 1}
Stosując odpowiednie kryterium zbadać zbieżność szeregu liczbowego
\[
    \text{a) } \sum_{n = 1}^{\infty}\frac{2^n \cdot 3^{2n - 1}}{4^{2n + 1}} \text{ oraz }
    \text{b) }\sum_{n = 1}^{\infty}(\frac{2n - 1}{2n + 3})^{3n^2}
\]

\subsubsection{Zadanie 2}
Stwierdzić (uzasadniając) czy szereg
\[
    \sum_{n = 1}^{\infty} (-1)^n \frac{3n + 1}{2n^2 - n}
\]
jest zbieżny bezwzględnie czy warunkowo.

\subsubsection{Zadanie 1 (rozwiązanie)}
W tym zadaniu we wszystkich kartkówkach jakie do tej pory przerabiałem zawsze trzeba było użyć raz
kryterium Cauchy'ego i raz d'Alemberta. Filipczak bardzo chciał byście napisali nazwę
kryterium z jakiego korzystacie i super by było jakbyście ją napisali poprawnie.
\begin{list}{}{}
    \item[a)]{
        W tym przypadku najlepiej będzie użyć \hyperlink{subsection.3.2}{kryterium d'Alemberta opisanego w sekcji 3.2}. Skąd taki wniosek?
        A stąd, że w podpunkcie b) będziemy korzystać z Cauchy'ego, a kiedyś musimy kryterium d'Alemberta użyć.
        Nasz przykład wyglądał następująco:
        \[
            \sum_{n = 1}^{\infty}\frac{2^n \cdot 3^{2n - 1}}{4^{2n + 1}}
        \]
        Przypomnijmy sobie szybko wzór na $g$. Warto zauważyć, że można ominąć wartość
        bezględną we wzorze ponieważ wszystkie wyrazy (czy jak to tam nazwać) szeregu są dodatnie:
        \[
            g = \lim_{n\to\infty}{\left\lvert \frac{a_{n + 1}}{a_n} \right\rvert}
            =  \lim_{n\to\infty}{ \frac{a_{n + 1}}{a_n} }
        \]
        Ja dla ułatwienia osobno upraszczam sobie $a^n$ oraz $a^{n + 1}$. Otrzymujemy więc:
        \[
            a^n = \frac{2^n \cdot 3^{2n - 1}}{4^{2n + 1}}=\frac{2^n \cdot 3^{2n} \cdot \frac{1}{3}}{4^{2n} \cdot 4}
        \]
        \[
            a^{n + 1} = \frac{2^{n + 1} \cdot 3^{2(n + 1) - 1}}{4^{2(n + 1) + 1}}=\frac{2^n \cdot 2 \cdot 3^{2n} \cdot 3}{4^{2n} \cdot 4^3}
        \]
        Podstawiając wartości te do wzoru na $g$ otrzymujemy:
        \[
            g =  \lim_{n\to\infty}{ \frac{a_{n + 1}}{a_n} }
            = \frac{2^n \cdot 2 \cdot 3^{2n} \cdot 3}{4^{2n} \cdot 4^3} \cdot \frac{4^{2n} \cdot 4}{2^n \cdot 3^{2n} \cdot \frac{1}{3}}
            = \frac{18}{16} = \frac{9}{8} > 1
        \]
        a więc na mocy d'Alemberta szereg jest rozbieżny.
    }
    \item[b)]{
        Jeśli w szeregu widzimy, że całość jest podniesiona do $n$'tej
        potęgi to najprawdopodobniej należy użyć \hyperlink{subsection.3.1}{kryterium Cauchy'ego opisanego w sekcji 3.1}.
        W podpunkcie w którym należy użyć tego kryterium ważne będzie również użycie \hyperlink{subsection.2.2}{wzoru na $e$ opisanego w sekcji 2.2}.
        Przypomnijmy sobie nasz przykład:
        \[
            \sum_{n = 1}^{\infty}(\frac{2n - 1}{2n + 3})^{3n^2}
        \]
        Kryterium Cauchy'ego stosujemy ponieważ nasza $n$'ta potęga pięknie nam się skraca z 
        $n$'tym pierwiastkiem. Tutaj również można zignorować wartość bezwzględną z tego samego powodu
        co w podpunkcie a).
        \[
            g = \lim_{n\to\infty}{\sqrt[n]{\left\lvert a_n \right\rvert}} = 
            \lim_{n\to\infty}{\sqrt[n]{(\frac{2n - 1}{2n + 3})^{3n^2}}}
            = \lim_{n\to\infty}(\frac{2n - 1}{2n + 3})^{3n}
        \]
        Tutaj odsyłam do \hyperlink{subsection.2.2}{sekcji z wzorem na $e$} ponieważ to jest za dużo pisania by
        krok po kroku to rozpisywać po raz kolejny. Mogę powiedzieć tylko, że:
        \[
            g = e^{-6} < 0
        \]
        więc na mocy kryterium Cauchy'ego szereg ten jest zbieżny.
    }
\end{list}
\newpage
\subsubsection{Zadanie 2 (rozwiązanie)}
W tym zadaniu trzeba rozpatrzyć zbieżność warunkową bądź bezwzględną w dwóch krokach:
\begin{list}{}{}
    \item[Krok 1.] {
        Sprawdzamy za pomocą \hyperlink{subsection.3.4}{kryterium Leibniza opisanego w sekcji 3.4} czy 
        dany szereg jest warunkowo zbieżny.

        Warunek 1: 
        \[
            \sum_{n = 1}^{\infty} (-1)^n \frac{3n + 1}{2n^2 - n} \text{, więc }
            a_n = \frac{3n + 1}{2n^2 - n}
        \]
        \[
            \lim_{a \to \infty}a_n = \lim_{a \to \infty}\frac{3n + 1}{2n^2 - n} =
            \lim_{a \to \infty}\frac{n^2(\frac{3}{n}+ \frac{1}{n^2})}{n^2(2 - \frac{1}{n})} = 0
        \]
        Warunek pierwszy spełniony.

        Warunek 2:
        W tym warunku jest trochę więcej pisania ale jest on giga prosty, każdy chyba
        umie dość biegle odejmować. Jak ktoś chce lepiej rozpisaną wersje to niech napiszę to
        pewnie dam. 
        \[
            a_n - a_{n + 1} = \frac{3n + 1}{2n^2 - n} - \frac{3n + 4}{2(n+1)^2 - n - 1} = 
            \frac{10n^2 + 10n + 1}{(4n - 1)(n^2 + n)} > 0
        \]
        To tak mniej więcej trzeba sobie ogarnąć, że jak podstawimy liczbę 1 lub większą to 
        liczba będzie większa od zera bo to co na górze i na dole jest większe od zera wiadomo o co chodzi.

        Jestem prawie pewny, że w każdym tym zadaniu szeregi te spełniają kryterium Leibniza
        co oznacza, że są zbieżne conajmniej warunkowo (ale i tak to trzeba udowodnić).
    }
    \item[Krok 2.]{
        Udawadnianie zbieżności bezwzględnej jest zadaniem bardzo prostym w którym mamy dowolność,
        jakiego kryterium możemy użyć ale we wszystkich przypadkach używałem 
        \hyperlink{subsection.3.3}{kryterium porównawczego opisanego w sekcji 3.3}. A więc 
        o co chodzi z tą zbieżnością bezwzględną. No musimy sobie obliczyć zbieżność dla wartości
        bezwzględnej wyrazów czy jakoś tak, czyli dla naszego przykładu
        \[
            \sum_{n = 1}^{\infty} (-1)^n \frac{3n + 1}{2n^2 - n}
        \]
        liczymy zbieżność dla szeregu
        \[
            \sum_{n = 1}^{\infty} \left\lvert(-1)^n \frac{3n + 1}{2n^2 - n}\right\rvert
            = \sum_{n = 1}^{\infty} \frac{3n + 1}{2n^2 - n}
        \]
        , więc jedyne co się zmienia to pozbywamy się tego $(-1)^n$. Jak już mówiłem, teraz kryterium
        jest dowolne ale najczęściej wystarczy kryterium porównawcze.
        \[
            \frac{3n + 1}{2n^2 - n} \simeq \frac{3n}{2n^2 - n} > \frac{1}{n}
        \]
        a więc z kryterium porównawczego wiemy, że szereg $\sum_{n = 1}^{\infty} \frac{3n + 1}{2n^2 - n}$ jest
        rozbieżny, więc nasz początkowy szereg nie jest zbieżny bezwzględnie.
    }
\end{list}
Odpowiedzią na to zadanie więc jest, że szereg jest zbieżny warunkowo. Gdyby 
w drugim kroku udowodniono, że jest zbieżny bezwzględnie to byśmy napisali, że jest
zbieżny bezwzględnie bo to nas bardziej interesuje.
\subsection{Kartkówka gr. A}
\subsubsection{Zadanie 1}
Stosując odpowiednie kryterium zbadać zbieżność szeregu liczbowego
\[
    \text{a) } \sum_{n = 1}^{\infty}(\frac{3n^2 - n}{3n^2 + 4n})^{n^2} \text{ oraz }
    \text{b) }\sum_{n = 1}^{\infty}\frac{(n!)^2 \cdot 3^{n+1}}{(2n + 1)!}
\]

\subsubsection{Zadanie 2}
Stwierdzić (uzasadniając) czy szereg
\[
    \sum_{n = 1}^{\infty} (-1)^{n+1} \frac{n + 2}{(2n + 1)!}
\]
jest zbieżny bezwzględnie czy warunkowo.
\subsubsection{Zadanie 1 (odp)}
Odpowiedź: 
\begin{list}{}{}
    \item[a) ] zbieżny
    \item[b) ] zbieżny
\end{list}
\subsubsection{Zadanie 2 (odp)}
Odpowiedź: zbieżny warunkowo

\subsection{Kartkówka gr. B}
\subsubsection{Zadanie 1}
Stosując odpowiednie kryterium zbadać zbieżność szeregu liczbowego
\[
    \text{a) } \sum_{n = 1}^{\infty}\frac{2 \cdot 4^{2n - 1}}{3^{2n + 1}} \text{ oraz }
    \text{b) }\sum_{n = 1}^{\infty}(\frac{n^2 - 1}{n^2 + 1})^{n^3}
\]

\subsubsection{Zadanie 2}
Stwierdzić (uzasadniając) czy szereg
\[
    \sum_{n = 1}^{\infty} (-1)^{n} \frac{2n - 1}{n^3 + 2n^2}
\]
jest zbieżny bezwzględnie czy warunkowo.
\subsubsection{Zadanie 1 (odp)}
Odpowiedź: 
\begin{list}{}{}
    \item[a) ] rozbieżny
    \item[b) ] zbieżny
\end{list}
\subsubsection{Zadanie 2 (odp)}
Odpowiedź: zbieżny bezwzględnie

\subsection{Kartkówka gr. A ale inna}
\subsubsection{Zadanie 1}
Stosując odpowiednie kryterium zbadać zbieżność szeregu liczbowego
\[
    \text{a) } \sum_{n = 1}^{\infty}\frac{3 \cdot 2^{2n - 1}}{5^{n + 1}} \text{ oraz }
    \text{b) }\sum_{n = 1}^{\infty}(\frac{3n - 1}{3n + 1})^{n^2}
\]

\subsubsection{Zadanie 2}
Stwierdzić (uzasadniając) czy szereg
\[
    \sum_{n = 1}^{\infty} (-1)^{n} \frac{n + 1}{3n^2 - n}
\]
jest zbieżny bezwzględnie czy warunkowo.
\subsubsection{Zadanie 1 (odp)}
Odpowiedź: 
\begin{list}{}{}
    \item[a) ] zbieżny
    \item[b) ] zbieżny
\end{list}
\subsubsection{Zadanie 2 (odp)}
Odpowiedź: zbieżny warunkowo
\subsection{UWAGA do odpowiedzi}
Te odpowiedzi liczył student debil (ja we własnej osobie) i nie daje gwarancji, że są one dobre.
Jeśli ktoś ma dowód, że się pomyliłem to będę wdzięczny jak mnie poinformuje, to będę mógł poprawić błąd.
Tyczy się to również błędów ortograficznych i innych takich, ponieważ notatkę robiłem sam, bez żadnych korekt
i innych tego typu podobnych. Każdy feedback jest dla mnie cenny, wielkie pozdro i powodzenia.

\end{document}